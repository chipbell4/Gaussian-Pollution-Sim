\documentclass[12pt]{article}

\title{Gaussian Pollution Model}
\author{Chip Bell}

\begin{document}
\maketitle
Many soot dispersion models use a Gaussian spread model. This model is very
similar, utilizing a Gaussian approach (modelling densities rather than
tracking individual particles). However, my approach is that wind disperses
soot particles in \emph{all} directions, but mostly in the direction of the
wind. Also, wind disperses soot a distance proportional to the wind speed.

Approaching this from a different perspective gives us the idea that the
pollution at given point is dependent on the speed of the wind at that given
area, along with the direction of the wind. So, some small area $dA$ receives
some fractional pollution from all of the area around it, depending on the
wind velocity and heading.

I believe that this dispersion is both Gaussian in distance (relating to wind
velocity) and in angle (relating to wind direction), giving us a two
dimensional Gaussian curve. The standard normal curve is as follows (CITE):
\[
	N(\mu, \sigma, x) = \frac{1}{\sqrt{2\pi}} e^{-\frac{(x-\mu)^2}{2\sigma^2} }
\]

We can extend this into a second dimension by simply multiplying the curve by
another normal curve (CITE). In our case, we want the curve to model wind
dispersion in a polar coordinate system $(r,\theta)$. We can choose two means
and two standard deviations, but those are simply parameters that can be
reasonably chosen through some experimentation and guessing.

Since, these are Gaussian distributions, we are guaranteed that the area
underneath the graphs (in our case, the volume) is 1. The pollution at some
specified location is the weighted sum of all pollution, where the weights are
determined by wind speed and direction. If we let $(a,b)$ be the point we seek
to evaluate, $d_{(x,y)}$ be the distance from the point $(x,y)$ to our point
$(a,b)$ and $\phi_{(x,y)}$ be the angle between our point $(a,b)$ and $(x,y)$,
we arrive at a method for determining the pollution density $P$ at some
location $(x,y)$ and time $t$.
\[
P(a, b, t+h) = \int_{R^2} N_r(\frac{v(t)}{h}, \sigma_r, d_{x,y})
	N_\theta(h(t), \sigma_\theta, \phi_{x,y}) P(x,y,t) dA 
\]
In this case, $v(t)$ is the velocity of the wind at the current time $t$, and
$h(t)$ is the heading of the wind at the current time. Since we are
integrating over all space, we are simply convoluting the ``pollution image''
in polar space.

However, an issue arises with using this method. Consider we are simulating
over some pixel space, say a standard XGA screen (resolution 1280 
$\times$ 768). For each of those pixels, every pixel on the screen must be
added into a weighted sum to calculate the pollution density for \emph{each
pixel}. This results in an $O(n^4)$ algorithm. Most modern computer screens 
operate at an even higher resolution, making speed an issue.

\end{document}
